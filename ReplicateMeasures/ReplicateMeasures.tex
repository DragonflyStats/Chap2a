\documentclass[Chap2dmain.tex]{subfiles}
\begin{document}
\newpage

\section{Exact description of repeated measures}
Bland and Altman strongly recommend the simultaneous estimation of repeatability and agreement by collecting replicated data (BA99pg16).
It is important first to clarify exactly what we mean when we refer to replicate observations. By replicates we mean two or more measurements on the same individual taken in identical conditions. In general this requirement means that the measurements are taken in quick succession.
One important feature of replicate observations is that they should be independent of each other. In essence, this is achieved by ensuring that the observer makes each measurement independent of knowledge of the previous value(s). This may be difficult to achieve in practice.
We do not usually expect second measurements of the same samples to differ systematically from first measurements. Indeed, such a systematic difference would indicate that the values were not true replicates.


Bland and Altman point out that a single measurement on each subject is not able to judge which method is more precise.

By replicates, Bland and Altman mean two or more measurements on the same individual taken in identical conditions taken in quick succession.

% Roy makes the assumption that these replicated measurements are equicorrelated.

Hamlett considers two types of multiple measurement; the linked case and the unlinked case.In the linked setting, the repeated measurement are linked in some way, ie taken over a series of days. In the unlinked setting, repeated measures are not linked

Carstensen's model assumes that replicate measurements are exchangeable within each method.

\end{document}