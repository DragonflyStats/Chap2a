\subsubsection{Algorithms for likelihood function optimization}Iterative numerical techniques are used to optimize the log-likelihood function and estimate the covariance parameters $\theta$. The procedure is subject to the constraint that $R$ and $D$ are both positive definite. The most common iterative algorithms for optimizing the likelihood function are the Newton-Raphson method, which is the preferred method, the expectation maximization (EM) algorithm and the Fisher scoring methods.

The EM algorithm, introduced by \citet{EM}, is an iterative technique for maximizing complicated likelihood functions. The algorithm alternates between performing an expectation (E) step
and the maximization (M) step. The `E' step computes the expectation of the log-likelihood evaluated using the current
estimate for the variables. In the `M' step, parameters that maximize the expected log-likelihood, found on the previous `E' step, are computed. These parameter estimates are then used to determine the distribution of the variables in the next `E' step. The algorithm alternatives between these two steps until convergence is reached.

The main drawback of the EM algorithm is its slow rate of
convergence. Consequently the EM algorithm is rarely used entirely in LME estimation,
instead providing an initial set of values that can be passed to
other optimization techniques.

The Newton Raphson (NR) method is the most common, and recommended technique for ML and
REML estimation. The NR algorithm minimizes an objective function defines as $-2$ times the log likelihood for the covariance parameters $\theta$. At every iteration the NR algorithm requires the
calculation of a vector of partial derivatives, known as the gradient, and the second derivative matrix with respect to the covariance parameters. This is known as the observed Hessian matrix. Due to the Hessian matrix, the NR algorithm is more time-consuming, but convergence is reached with fewer iterations compared to the EM algorithm. The Fisher scoring algorithm is an variant of the NR algorithm that is more numerically stable and likely to converge, but not recommended to obtain final estimates.
