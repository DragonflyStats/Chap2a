
\section{Variations of the Bland-Altman Plot} Referring to the
assumption that bias and variability are constant across the range
of measurements, \citet{BA99} address the case where there is an
increase in variability as the magnitude increases. They remark
that it is possible to ignore the issue altogether, but the limits
of agreement would wider apart than necessary when just lower
magnitude measurements are considered. Conversely the limits would
be too narrow should only higher magnitude measurements be used.
To address the issue, they propose the logarithmic transformation
of the data. The plot is then formulated as the difference of
paired log values against their mean. Bland and Altman acknowledge
that this is not easy to interpret, and may not be suitable in
all cases.

\citet{BA99} offers two variations of the Bland-Altman plot that
are intended to overcome potential problems that the conventional
plot would inappropriate for. The first variation is a plot of
case-wise differences as percentage of averages, and is
appropriate when there is an increase in variability of the
differences as the magnitude increases. The second variation is a
plot of case-wise ratios as percentage of averages. This will
remove the need for $log$ transformation. This approach is useful
when there is an increase in variability of the differences as the
magnitude of the measurement increases. \citet{Eksborg} proposed
such a ratio plot, independently of Bland and Altman.
\citet{Dewitte} commented on the reception of this article by
saying `Strange to say,this report has been overlooked'.

\end{document
