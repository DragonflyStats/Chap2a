
Model I regression [Criterion v Test]
[Cornbleet Gochman 1979] define this analysis as the case in which the independent variable, X, is measured without error, with y as the dependent variable.
 
In method comparison studies, the X variable is a precisely measured reference method. In the [Cornbleet Gochman 1979] paper. It is argued that criterion may be regarded as the correct value. Other papers dispute this.
 
 
Model II regression [Test V Test]
In this type of analysis,both of the measurement methods are test methods, with both expected to be subject to error. Deming regression is an approach to model II regression.

Deming Regression

 

The 95% confidence interval for the Intercept can be used to test the hypothesis that A=0. This hypothesis is accepted if the confidence interval for A contains the value 0. If the hypothesis is rejected, then it is concluded that A is significant different from 0 and both methods differ at least by a constant amount. 

 

The 95% confidence interval for the Slope can be used to test the hypothesis that B=1. This hypothesis is accepted if the confidence interval for B contains the value 1. If the hypothesis is rejected, then it is concluded that B is significant different from 1 and there is at least a proportional difference between the two methods. Cochrane Cornbleet
The authors make the distinction between model I and model II regression types.
Model II regression is the appropriate type when the predictor variable “x” is measured with imprecision.
Cornbleet and Cochrane remark that clinical laboratory measurements usually increase in absolute imprecision when larger values are measured.[**]
Guidelines
Always plot the data. Suspected outliers may be identified from the scatter plot.
S_{ex}  represents the precision of a single x measurement near the mean value of X
\lambda = frac{S^2_{ex}}{S^2_{ey}}



