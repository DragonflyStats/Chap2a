%--------------------------------------------------------------------%
\subsection{Bland and Altman 1999}
As noted by Bland and Altman 1999, the repeatability of two methods of measurement can  potentially limit
Repeatability (using Bland-Altman plot)
The Bland-Altman plot may also be used to assess a method’s repeatability by comparing repeated measurements using one single measurement method on a sample of items.
The plot can then also be used to check whether the variability or precision of a method is related to the size of the characteristic being measured.
Since for the repeated measurements the same method is used, the mean difference should be zero.
Therefore the Coefficient of Repeatability (CR) can be calculated as 1.96 (often rounded to 2) times the standard deviation of the case-wise differences.
%--------------------------------------------------------------------%
\subsection{Wikipedia Entry (Re-written)}
Repeatability (or test-retest reliability[1])  describes the variation in measurements taken by a single method of measurement on the same item and under the same conditions. 
A less-than-perfect test-retest reliability causes test-retest variability. Such variability can be caused by, for example, intra-individual variability and intra-observer variability. 
A measurement may be said to be repeatable when this variation is smaller than some agreed limit.
Test-retest variability is practically used, for example, in medical monitoring of conditions. In these situations, there is often a predetermined "critical difference", and for differences in monitored values that are smaller than this critical difference, the possibility of pre-test variability as a sole cause of the difference may be considered in addition to, for examples, changes in diseases or treatments.
According to the Guidelines for Evaluating and Expressing the Uncertainty of NIST Measurement Results, the following conditions need to be fulfilled in the establishment of repeatability:
•	the same measurement procedure
•	the same observer
•	the same measuring instrument, used under the same conditions
•	the same location
•	repetition over a short period of time.
%--------------------------------------------------------------------%
\subsection{BXC Book (chapter 9)}
The assessment of method-specific repeatability and reproducibility is of interest in its own right.
Repeatability and reproducibility can only be assessed when replicate measurements by each method are available.
If replicate measurements by a method are available, it is simple to estimate the measurement error for a method, using a model with fixed effects for item, then taking the residual standard deviation as measurement error standard deviation.
However, if replicates are linked, this may produce an estimate that biased upwards.
The repeatability coefficient (or simply repeatability) for a method is defined as the upper limit of a
prediction interval for the absolute difference between two measurements by the same method on the same
item under identical circumstances (see above conditions)

\[y_{mir}  = \alpha_{m} + \beta_m( \mu_i + a_{ir} + c_{mi}) + e_{mir}\]

The variation between measurements under identical circumstances.

\end{document}
