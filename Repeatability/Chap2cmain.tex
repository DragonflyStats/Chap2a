\documentclass[12pt, a4paper]{article}
\usepackage{natbib}
\usepackage{vmargin}
\usepackage{graphicx}
\usepackage{epsfig}
\usepackage{subfigure}
%\usepackage{amscd}
\usepackage{subfiles}
\usepackage{amssymb}
\usepackage{amsbsy}
\usepackage{amsthm, amsmath}
%\usepackage[dvips]{graphicx}
\bibliographystyle{chicago}
\renewcommand{\baselinestretch}{1.8}
% left top textwidth textheight headheight % headsep footheight footskip
\setmargins{3.0cm}{2.5cm}{15.5 cm}{23.5cm}{0.5cm}{0cm}{1cm}{1cm}
\pagenumbering{arabic}

%------------------------------------------------------------------------------%
\begin{document}
\author{Kevin O'Brien}
\title{Repeatability}
\date{\today}
\maketitle
\tableofcontents

\subfile{Repeatability}
\subfile{RepeatabilityCoefficient}
%----------------------------------------------------------------------------------------%
\newpage


%------------------------------------------------------------------------------%
\subsection{Relevance of Repeatability} 
Repeatability of two method limit the amount of agreement that is possible.\\
If one method has poor repeatability, the agreement is bound to be poor. If both methods have poor repeatability, agreement is
even worse.


%----------------------------------------------------------------------------------------%
\section{Repeatability}
\subsection{Repeatability and gold standards}
Currently the phrase `gold standard' describes the most accurate method of measurement available. No other criteria are set out. Further to \citet{dunnSEME}, various gold standards have a varying levels of repeatability. Dunn cites the example of the sphygmomanometer, which is prone to measurement error. Consequently it can be said that a measurement method can be the `gold standard', yet have poor repeatability. Some authors, such as [cite] and [cite] have recognized this problem. Hence, if the most accurate method is considered to have poor repeatability, it is referred to as a 'bronze standard'.  Again, no formal definition of a 'bronze standard' exists.

The coefficient of repeatability may provide the basis of formulation a formal definition of a `gold standard'. For example, by determining the ratio of $CR$ to the sample mean $\bar{X}$. Further to [Lin], it is preferable to have a sample size specified in advance. A gold standard may be defined as the method with the lowest value of $\lambda = CR /\bar{X}$ with $\lambda < 0.1\%$. Similarly, a silver standard may be defined as the method with the lowest value of $\lambda $ with $0.1\% \leq \lambda < 1\%$. Such thresholds are solely for expository purposes.


%----------------------------------------------------------------------------------------%

\chapter{Ceofficient of Repeatability}
 %Definition of Repeatability


\section{Add Ins}
importance of repeatability
'cursiously replicate measurements are rarely made in method comparison studies, so that an important aspect of comparability is 
often overlooked.

lack of repaeatability can interfere with the comparsion of two methods because if one methods has poor repeatability, in the sense that there is
considerable variation in repeated measurements on the same subject, the agreement between two methods is bound to be porr.



%------------------------------------------------------------------------------------------%
\section{Bland and Altman}
\begin{itemize}
\item Two readings by the same method will be within $1.96
\sqrt{2} \sigma_w $ or $2.77 \sigma_w $ for 95\% of subjects. Thisvalue is called the repeatability coefficient.

\item For observer J using the sphygmomanometer $ \sigma_w = \sqrt{37.408} = 6.116$ and so the repeatability coefficient is
$2:77 \times 6.116 = 16:95$ mmHg.

\item For the machine S,$ \sigma_w = \sqrt{83.141} = 9.118$ and the repeatability coefficient is $2:77 \times 9.118 = 25.27$ mmHg.

\item Thus, the repeatability of the machine is 50\% greater than that of the observer.
\end{itemize}
%-------------------------------------------------------------------%
\section{Carstensen}
\begin{itemize}
\item The limits of agreement are not always the only issue of
interest — the assessment of method specific repeatability and
reproducibility are of interest in their own right.

\item Repeatability can only be assessed when replicate
measurements by each method are available.

\item The repeatability coefficient for a method is defined as the
upper limits of a prediction interval for the absolute difference
between two measurements by the same method on the same item under
identical circumstances.

\item If the standard deviation of a measurement is $\sigma$ the
repeatability coefficient is $2\times\sqrt{2} \sigma = 2.83\times
\sigma \approx 2.8 \sigma$.


\item The repeatability of measurement methods is calculated
differently under the two models \item Under the model assuming
exchangeable replicates (1), the repeatability is based only on
the residual standard deviation, i.e. $2.8\sigma_m$


\item Under the model for linked replicates (2) there are two
possibilities depending on the circumstances.

\item If the variation between replicates within item can be
considered a part of the repeatability it will be $2.8 \sqrt{
\omega^2 + \sigma^2_m}$.

\item However, if replicates are taken under substantially
different circumstances, the variance component $\omega^2$ may be
considered irrelevant in the repeatability and one would therefore
base the repeatability on the measurement errors alone, i.e. use
$2.8 \sigma_m$.


\end{itemize}
\newpafe



%------------------------------------------------------------------------------%
\subsection{Repeatability}



\citet{BA99} strongly recommends the simultaneous estimation of
repeatability and agreement be collecting replicated data.
\citet{ARoy2009} notes the lack of convenience in such
calculations.


If one method has poor repeatability in the sense of considerable
variability, then agreement between two methods is bound to be
poor \citep{ARoy2009}.

It is important to report repeatability when assessing
measurement, because it measures the purest form of random error
not influenced by other factors \citep{Barnhart}.

\end{document} 
