\documentclass[Chap2cmain.tex]{subfiles}
\begin{document}

\section{Reproducibility}
 
It is advisable to be able to distinguish between Repeatability and a similar concept ‘Reproducibility’. Reproducibility is
 
\subsection{2 The Coefficient of Repeatability}
Since for the repeated measurements the same method is used, the mean difference should be zero.
 
Therefore the Coefficient of Repeatability (CR) can be calculated as 1.96 (or 2) times the standard deviations of the differences between the two measurements (d2 and d1):
WRONG

\newpage
\section{Repeatability}

The quality of repeatability is the ability of a measurement method to give consistent results for a particular subject. That is to say that a measurement will agree with prior and subsequent measurements of the same subject.

Repeatability is defined by the \citet{IUPAC} as `the closeness of agreement between independent results obtained with the same method on identical test material, under the same conditions (same
operator, same apparatus, same laboratory and after short intervals of time)'  and is determined by taking multiple measurements on a series of subjects.

Repeatability is important in the context of method comparison because the repeatability of two methods influence the amount of agreement which is possible between those methods. If one method have poor repeatability, then agreement with that method and another will necessarily be poor also.
\citet{barnhart} and \citet{roy} highlight the importance of reporting repeatability in method comparison, because it measures the purest random error not influenced by any external factors. Statistical procedures on within-subject variances of two methods are equivalent to tests on their respective repeatability coefficients. A formal test is introduced by \citet{roy}, which will discussed in due course.



%------------------------------------------------------------------------------%
\section{Repeatability}
A measurement method can be said to have a good level of repeatability if there is consistency in repeated measurements on
the same subject using that method. Conversely, a method has poor
repeatability if there is considerable variation in repeated measurements.


This is relevant to method comparison studies because the 'repeatabilities' of the two methods of measurement affects the
level of agreement of those methods.Poor repeatability in one method would result in poor agreement. More so if there is poor
repeatability in both methods.

The British standards Insitute[1979] define a coefficient of repeatability  as \emph{the value below which the difference
between two single test results..may be expected to lie within a specified probability.}In the absence of
other indications, the probability is 95\%.


\end{document}
