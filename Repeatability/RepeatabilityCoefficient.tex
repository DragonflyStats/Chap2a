\documentclass[Chap2cmain.tex]{subfiles}
\begin{document}

\newpage
%---------------------------------------------%
\section{Coefficient of Repeatability}
\subsection{Repeatability}
Barnhart emphasizes the importance of repeatability as part of an overall method comparison study. Before there can be good agreement between two methods, a method must have good agreement with itself. The coefficient of repeatability , as proposed by \citet{BA99} is an important feature of both Carstensen's and Roy's methodologies. The coefficient is calculated from the residual standard deviation (i.e. $1.96 \times \sqrt{2} \times \sigma_m$ = $2.83 \sigma_m$).


The coefficient of repeatability is a measure of how well a
measurement method agrees with itself over replicate measurements
\citep{BA99}. Once the within-item variability is known, the
computation of the coefficients of repeatability for both methods
is straightforward.

%------------------------------------------------------------------------------%
\subsection{Coefficient of Repeatability}
The Bland Altman Method offers the analyst a measurement on the repeatability of the methods.\\ The Coefficient of Repeatability
(CR) can be calculated as 1.96 (or 2) times the standard deviations of the differences between the two measurements (d2 and
d1).
\subsection{Note 1: Coefficient of Repeatability}
The coefficient of repeatability is a measure of how well a
measurement method agrees with itself over replicate measurements
\citep{BA99}. Once the within-item variability is known, the
computation of the coefficients of repeatability for both methods
is straightforward.
\newpage
The Bland Altman method offers the analyst a measurement on the repeatability of the methods.\\ The Coefficient of Repeatability
(CR) can be calculated as 1.96 (or 2) times the standard deviations of the differences between the two measurements.


The coefficcient of repeatability is a measure of how well a measurement method agrees
with itself over replicate measurements (Bland and Altman, 1999). 
 
 Once the within-item variability is known, the computation of the coefficients of repeatability for both
 methods is straightforward.
%------------------------------------------------------------------------------%

\subsection{Repeatability coefficient}
\citet{BA99} introduces the repeatability coefficient for a method, which is defined as the upper limits of a prediction interval for the absolute difference between two measurements by the same
method on the same item under identical circumstances \citep{BXC2008}.

$\sigma^2_{x}$ is the within-subject variance of method $x$. The repeatability coefficient is $2.77 \sigma_{x}$ (i.e. $1.96 \times \sqrt{2} \sigma_{x}$). For $95\%$ of subjects, two replicated measurement by the same method will be within this repeatability coefficient.

\newpage

%-----------------------------------------------------------------------------------------------------%
\newpage

\subsection{Repeatability}
Barnhart emphasizes the importance of repeatability as part of an overall method comparison study. Before there can be good agreement between two methods, a method must have good agreement with itself. The coefficient of repeatability , as proposed by \citet{BA99} is an important feature of both Carstensen's and Roy's methodologies. The coefficient is calculated from the residual standard deviation (i.e. $1.96 \times \sqrt{2} \times \sigma_m$ = $2.83 \sigma_m$).
%---------------------------------------------%
\end{document}



