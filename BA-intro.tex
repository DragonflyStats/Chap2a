\chapter{The Bland Altman Plot}
\section{Bland Altman Plots}
The issue of whether two measurement methods are comparable to the
extent that they can be used interchangeably with sufficient
accuracy is encountered frequently in scientific research.
Historically comparison of two methods of measurement was carried
out by use of matched pairs correlation coefficients or simple
linear regression. Bland and Altman recognized the inadequacies of
these analyses and articulated quite thoroughly the basis on which
of which they are unsuitable for comparing two methods of
measurement \citep*{BA83}.

As an alternative they proposed a simple statistical methodology
specifically appropriate for method comparison studies. They
acknowledge that there are other valid methodologies, but argue
that a simple approach is preferable to complex approaches,
\emph{"especially when the results must be explained to
non-statisticians"} \citep*{BA83}.

The first step recommended which the authors argue should be
mandatory is construction of a simple scatter plot of the data.
The line of equality ($X=Y$) should also be shown, as it is
necessary to give the correct interpretation of how both methods
compare. A scatter plot of the Grubbs data is shown in figure 2.1.
A visual inspection thereof confirms the previous conclusion that
there is an inter method bias present, i.e. Fotobalk device has a
tendency to record a lower velocity.



In light of shortcomings associated with scatterplots,
\citet*{BA83} recommend a further analysis of the data. Firstly
differences of measurements of two methods on the same subject
should  be calculated, and then the average of those measurements
(Table 1.1). The averages of the two measurements is considered by
Bland and Altman to the best estimate for the unknown true value.
Importantly both methods must measure with the same units. These
results are then plotted, with differences on the ordinate and
averages on the abscissa (figure 1.2). \citet*{BA83}express the
motivation for this plot thusly:
\begin{quote}
"From this type of plot it is much easier to assess the magnitude
of disagreement (both error and bias), spot outliers, and see
whether there is any trend, for example an increase in
(difference) for high values. This way of plotting the data is a
very powerful way of displaying the results of a method comparison
study."
\end{quote}
\newpage
% latex table generated in R 2.6.0 by xtable 1.5-5 package
% Thu Aug 27 16:31:52 2009
\begin{table}[tbh]
\begin{center}

\begin{tabular}{|c|c|c|c|c|}
  \hline
 Round & Fotobalk [F] & Counter [C] & Differences [F-C] & Averages [(F+C)/2] \\
  \hline
1 & 793.80 & 794.60 & -0.80 & 794.20 \\
  2 & 793.10 & 793.90 & -0.80 & 793.50 \\
  3 & 792.40 & 793.20 & -0.80 & 792.80 \\
  4 & 794.00 & 794.00 & 0.00 & 794.00 \\
  5 & 791.40 & 792.20 & -0.80 & 791.80 \\
  6 & 792.40 & 793.10 & -0.70 & 792.80 \\
  7 & 791.70 & 792.40 & -0.70 & 792.00 \\
  8 & 792.30 & 792.80 & -0.50 & 792.50 \\
  9 & 789.60 & 790.20 & -0.60 & 789.90 \\
  10 & 794.40 & 795.00 & -0.60 & 794.70 \\
  11 & 790.90 & 791.60 & -0.70 & 791.20 \\
  12 & 793.50 & 793.80 & -0.30 & 793.60 \\
   \hline
\end{tabular}
\caption{Fotobalk and Counter Methods: Differences and Averages}
\end{center}
\end{table}




\subsection{Repeated Measurements }
In cases where there are repeated measurements by each of the two
methods on the same subjects , Bland Altman suggest calculating
the mean for each method on each subject and use these pairs of
means to compare the two methods.
\\
The estimate of bias will be unaffected using this approach, but
the estimate of the standard deviation of the differences will be
too small, because of the reduction of the effect of repeated
measurement error. Bland Altman propose a correction for this.
\\
Carstensen attends to this issue also, adding that another
approach would be to treat each repeated measurement separately.

\subsection{Criticism of Bland Altman Plot}
Hopkins[$8$] argues that the plot indicates incorrectly that there
are systematic differences or bias in the relationship between two
measures, when one has been calibrated against the other.
\\
An Evaluation of the correlation between the difference and means
complement the analysis.
\\
Bland and Altman caution that the calculations are based on the
assumption that the data is normally distributed. This can be
verified by using a histogram. If Data is not normally
distributed, it can be transformed.









\chapter{REGRESSION}%%%%%%%%%%%%%%%%%%%%%%%%%%%%%%%%%%%%%%%%%%%%%%%%%%%%%%%%%%%%%%%%%%%%%%%%%%%%%%%%%%%%%%%%%%%%%%%%% Regression
\section{Model II Regression}
\subsection{Simple Linear Regression} Simple Linear Regression is  well
known statistical technique , wherein estimates for slope and
intercept of the line of best fit are derived according to the
Ordinary Least Square (OLS) principle.This method is known to
Cornbleet and Cochrane as Model I regression.
\\
\\
In Model I regression, the independent variable is assumed to be
measured without error. For method comparison studies, both sets
of measurement must be assumed to be measured with imprecision and
neither case can be taken to be a reference method. Arbitrarily
selecting either method as the reference will yield two
conflicting outcomes. A fitting based on '$X$ on $Y$' will give
inconsistent results with a fitting based on '$Y$ on $X$'.
Consequently model I regression is inappropriate for such cases.
\\
\\
Conversely, Cornbleet Cochrane state that when the independent
variable $X$ is a precisely measured reference method, Model I
regression may be considered suitable. They qualify this statement
by referring the $X$ as \emph{the 'correct' value}, tacitly
implying that there must still be some measurement error present.
The validity of this approach has been disputed elsewhere.




\subsection{Model II regression}
Cochrane and Cornbleet argue for the use of methods that based on
the assumption that both methods are imprecisely measured ,and
that yield a fitting that is consistent with both '$X$ on $Y$' and
'$Y$ on $X$' formulations. These methods uses alternatives to the
OLS approach to determine the slope and intercept.
\\
They describe three such alternative methods of regression; Deming
, Mandel, and Bartlett regression. Collectively the authors refer
to these approaches as Model II regression techniques.

%%%%%%%%%%%%%%%%%%%%%%%%%%%%%%%%%%%%%%%%%%%%%%%%%%%%%%%%%%%%%%%%%%%%%%%%%

\subsection{Distribution of Maxima} It is possible to use Order
Statistics theory to assess conditional probabilities. With two
random variables $T_{0}$ and $T_{1}$, we define two variables $Z$
and $W$ such that they take the maximum and minimum values of the
pair of $T$ values.\subsection{Plot of the Maxima against the
Minima}


In Figure 1,  The Maximas are plotted against their corresponding
minima. The Critical values of the Maxima and Minima are displayed
in the dotted lines.The Line of Equality depicts the obvious
logical constraint of the each Maximum value being greater than
its corresponding minimum value.



The scientific question at hand is the correct approach to
assessing whether two methods can be used interchangeably.
\citet{BA99} expresses this as follows:
\begin{quote}We want to
know by how much (one) method is likely to differ from the
(other), so that if it not enough to cause problems in the
mathematical interpretation we can ... use the two
interchangeably.
\end{quote}



Consequently, of the categories of method comparison study,
comparison studies, the second category, is of particular
importance, and the following discussion shall concentrate upon
it. Less emphasis shall be place on the other three categories.

 \bigskip Further to \citet{BA86}, 'equivalence' of two methods expresses
 that both can be used interchangeably.
\citet[p.49]{DunnSEME} remarks that this is a very restrictive
interpretation of equivalence, and that while agreement indicated
equivalence, equivalence does not necessarily reflect agreement.

The main difference between Myers proposed method and the Bland
Altman is that the random effects model is used to estimate the
within-subject variance after adjusting for known and unknown
variables. The Bland Altman approach uses one way analysis of
variance to estimate the within subject variance. In general, the
random effects model is an extension of the analysis of the ANOVA
method and it can adjust for many more covariates than the ANOVA
method



\subsection{Criticism of Bland Altman Plots}

An Evaluation of the correlation between the difference and means
complement the analysis.
\\
Bland and Altman caution that the calculations are based on the
assumption that the data is normally distributed. This can be
verified by using a histogram. If Data is not normally
distributed, it can be transformed.
\\
Luiz \emph{et al} remarks that that Bland Altman Plot should be
used only for small data sets, as the use of an index will be of
little value to the analysis.


