\documentclass[Chap2main.tex]{subfiles}

% Load any packages needed for this document

\begin{document}



\section{Formal Testing}
The Bland-Altman plot is a simple tool for inspection of the data,
but in itself it offers no formal testing procedure in this
regard. To this end, the approach proposed by \citet{BA83} is a
formal test on the Pearson correlation coefficient  of casewise
differences and means ($\rho_{AD}$). According to the authors,
this test is equivalent to a well established tests for equality
of variances, known as the `Pitman Morgan Test' \citep{Pitman,
	Morgan}.

For the Grubbs data, the correlation coefficient estimate
($r_{AD}$) is 0.2625, with a 95\% confidence interval of (-0.366,
0.726) estimated by Fishers 'r to z' transformation \citep{Cohen}.
The null hypothesis ($\rho_{AD}$ =0) would fail to be rejected.
Consequently the null hypothesis of equal variances of each method
would also fail to be rejected.

There has no been no further mention of this particular test in
the subsequent article published by Bland and Altman, although
\citet{BA99} refers to Spearmans' rank correlation coefficient.

\end{document}
